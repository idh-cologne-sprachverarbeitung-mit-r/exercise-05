\documentclass[aspectratio=169]{beamer}

\usepackage[english]{babel}
\usepackage{outlines}
\usepackage{csquotes}
\usepackage[style=authoryear-comp]{biblatex}
\usepackage{slidecite}
\usepackage{tikz}
\usepackage{soul}
\usepackage{fontawesome}
\usepackage{lingmacros}
\usepackage{bbding}
\usepackage{booktabs}
\usepackage[normalem]{ulem}
\usepackage{siunitx}
\usepackage{tikzsymbols}
\usepackage{dcolumn}
\usepackage{multirow}
\usepackage{wasysym}
\usepackage{pifont}
\usepackage{amsmath}
\usepackage{mathtools}
\usepackage{bbding}
\usepackage{listings}


\addbibresource{lit.bib}

\usetikzlibrary{calc,positioning}
\usetikzlibrary{arrows}
\usetikzlibrary{fit}
\usetikzlibrary{intersections}
\usetikzlibrary{trees}
\usetikzlibrary{arrows}
\usetikzlibrary{shapes.misc}
\usetikzlibrary{shapes.geometric}
\usetikzlibrary{shapes.symbols}
\usetikzlibrary{shapes.callouts}
\usetikzlibrary{shapes.arrows}
\usetikzlibrary{shapes.multipart}
\usetikzlibrary{decorations.text}
\usetikzlibrary{matrix}

\usefonttheme{professionalfonts}
\setbeamertemplate{navigation symbols}{}

\title{Week 5: Functions}
\subtitle{Quantitative Textanalyse mit R}
\author{Nils Reiter}

\lstset{basicstyle=\small\ttfamily, % print whole listing small 
	keywordstyle=\color{blue}, % underlined bold black keywords 
	%identifierstyle=\color{brown}, % nothing happens 
	commentstyle=\color{olive},
	stringstyle=\color{red}, % typewriter type for strings 
	showstringspaces=false,
	morekeywords={function}}

\begin{document}
\pgfdeclarelayer{background} 
\pgfdeclarelayer{foreground}
\pgfsetlayers{background,main,foreground}

\frame{\maketitle}

\begin{frame}[fragile]{R so far}
\begin{outline}
\1 Vector, list, matrix, data.frame
\2 Vectors are 1-based (i.e., start with 1)
\1 Types: numeric, character, logical
\1 Expressions: Assignments of values to variable names
\1 Selecting elements in containers: [ ]
\end{outline}
\pause
\begin{lstlisting}[language=r]
(1:100)[c(TRUE,FALSE)] # every second element
m <- matrix(1:100, nrow=25)
m[4,5] # row 4 col 5
m[4,]  # row 4
m[,5]  # col 5
\end{lstlisting}
\end{frame}


\begin{frame}[fragile]{Control structures}
\begin{block}{Grouped expressions: \lstinline!{ expr_1, expr_2, ..., expr_n }!}
\begin{outline}
\1 Commands can be grouped with curly braces
\1 The value of the whole group is the value of the last expression within the group
\1 Grouped expressions do not form a scope on their own
\end{outline}
\end{block}
\pause
\begin{block}{Conditional execution: \lstinline!if (expr_1) expr_2 else expr_3!}
\begin{outline}
\1 \lstinline{expr_2} only executed if \lstinline{expr_1} evaluates to a single logical TRUE value
\1 Implicit use of \lstinline{as.logical(expr_1)[1]}
\1 Warning if \lstinline{length(expr_1)>1}
\end{outline}
\end{block}
\end{frame}

\begin{frame}[fragile]{Control structures}
\begin{block}{Loops: \lstinline!for (name in expr) expr_2 !}
\begin{outline}
\1 \lstinline!expr_2! repeatedly evaluated, while \lstinline!name! is set to the elements of \lstinline!expr!
\1 In R, no difference between `foreach' and `for'
\1 \lstinline!for i in 1:100 print(i)! shows their equivalence
\pause
\1 Loops are used \emph{much less} in R compared to other languages
\2 Most operations work in a \emph{vectorized} fashion: We call a function that gets executed to every element of the vector
\2 This vectorization is much faster, because it can be processed en bloc
\2 Rule of thumbs: Avoid writing for-loops
\2 There is even a vectorized version of if: \lstinline{ifelse(condition,a,b)}
\end{outline}
\end{block}
\end{frame}

\section{Functions}

\begin{frame}[fragile]{Functions}
\begin{outline}
\1 Functions are mini-programs
\1 Named collections of commands
\1 Similar to other programming languages
\1 Syntax
\2 Function definition: \lstinline!name <- function(arg_1, arg_2, ...) expression!
\2 Function call: \lstinline!name(arg_1, arg_2, ...)!
\pause
\2 If the name starts and ends with \%, and the function takes two arguments, it can be used as a binary operator \\
\begin{lstlisting}
"%add%" <- function(x,y) {sum(x,y)}
5 %add% 7 # returns 12
\end{lstlisting} 
\end{outline}
\end{frame}

\begin{frame}[fragile]{Example}
\begin{outline}
\1 Accuracy: Percentage of correctly classified instances
\1 Input: Two lists of labels
\1 Output: Number
\end{outline}
\begin{lstlisting}
accuracy <- function(gold, system) {
    # let's do that together
}
\end{lstlisting}
\end{frame}

\begin{frame}[fragile]{Example}
\begin{outline}
\1 Accuracy: Percentage of correctly classified instances
\1 Input: Two lists of labels
\1 Output: Number
\end{outline}
\begin{lstlisting}
accuracy <- function(gold, system)
    sum(gold == system)/length(gold)
\end{lstlisting}
Any remaining issues? What could go wrong?
\end{frame}

\begin{frame}{Example}{Edge cases}
\begin{outline}
\0 Not the same length
\1 We should verify that the vectors have the same length
\0 Different types
\1 Types are coerced to character if necessary
\2 Depending on use case: Maybe issue a warning
\0 Different factors
\1 Not a problem, factor levels can be compared as character vectors
\end{outline}
\end{frame}

\begin{frame}[fragile]{Named Arguments}
\begin{outline}
\1 Function arguments can be named
\1 Named arguments can be given in any order
\2 If both named and positional arguments are mixed, positional arguments go first
\1 Using named arguments increases readability of your code \\
\lstinline!accuracy(system=listOfLabels, gold=realLabels)
\end{outline}
\end{frame}

\begin{frame}[fragile]{Default Values}
\begin{outline}
\1 Specified in the function definition
\begin{lstlisting}
accuracy <- function(gold, system, times=100)
    times * (sum(gold == system)/length)
\end{lstlisting}
\1 Can be defined based on other arguments (!)
\begin{lstlisting}
accuracy <- function(gold, system, limit=1:length(gold))
    sum(gold[limit] == system[limit])/length(limit)
\end{lstlisting}
\end{outline}
\end{frame}

\begin{frame}[fragile]{Passing on arguments}
\begin{outline}
\1 Sometimes we want to pass on arguments to other functions, in particular if we write `wrapper functions' (i.e., functions that wrap another function)
\1 In these cases, we can use the \lstinline!...! argument (yes, it's three regular dots)
\1 All arguments not defined by our function are passed on to another function
\end{outline}
\begin{lstlisting}
partOfRandomVector <- function(n, x, ...) {
    v <- runif(n, ...)
    v[v<x]
}
\end{lstlisting}
We can now use all (named) arguments of \lstinline!runif! as arguments for \lstinline!partOfRandomVector! without further declaration
\end{frame}

\begin{frame}[fragile]{Anonymous functions (and apply)}
\begin{outline}
\1 Similar to passing literal arguments to functions (as we have done before), we can pass anonymous functions
\1 They look the same, but are not assigned to a name: \lstinline!function(arg_1, arg_2) expression!
\1 The function \lstinline!apply()! expects a function as third argument
\2 \lstinline!apply()! applies this function to each row or column of a given matrix
\end{outline}
\begin{example}
\begin{lstlisting}
m <- matrix(1:100, ncol=5)
apply(m, 1, function(x) { sum(x**2) } )
\end{lstlisting}
\end{example}
\end{frame}

\section{Exercise}

\begin{frame}{Exercise 5}
\begin{outline}
\1 At the usual place: \url{https://github.com/idh-cologne-sprachverarbeitung-mit-r/exercise-05}
\2 Includes these slides
\1 Less exercises, but a bit more complicated
\end{outline}
\end{frame}

\begin{frame}[fragile]{ATM}
\begin{outline}
\1 When we withdraw money from an ATM, it has to decide which bills it gives
\1 We will implement this function in R
\1 The function \lstinline!atm(v)! takes a numeric value and returns a named list
\2 The names are the bill types (i.e., 50)
\2 The values are the number of bills to give us
\1 How to decide?
\2 We want as few bills as possible (this is not what real ATMs to \dots)
\end{outline}
\pause
\begin{example}
\begin{lstlisting}
> unlist(atm(50))
500 200 100  50  20  10   5 
  0   0   0   1   0   0   0 
> unlist(atm(195))
500 200 100  50  20  10   5 
  0   0   1   1   2   0   1 
\end{lstlisting}
\end{example}
\end{frame}

\begin{frame}{Fleiss' Kappa}
\begin{outline}
\1 Fleiss' Kappa is a metric to calculate inter-annotator agreement
\begin{tabular}{lcccc}
\toprule
Word & A & B & C & D \\
\midrule
1 & 2 & 1 &   &   \\
2 &   &   & 1 & 2 \\
3 & 1 & 1 &   & 1 \\
4 & 3 &   &   &   \\
\bottomrule
\end{tabular}
\1 Observed agreement: How many agreements have been achieved?
\1 Expected agreement: How many agreements can be expected if everyone guesses?
\end{outline}
\end{frame}

\begin{frame}{Inter-Annotator Agreement}{Observed Agreement}

\begin{block}{Normalized observed agreement for item $i$}
Problem: $k$ categories, $n$ annotators, $N$ items 
\uncover<2->{\begin{equation*}\only<4->{\hat P_i = }\uncover<3->{\alert<3>{\underbrace{\frac{1}{n(n-1)}}_{\textnormal{Scaling for annotators}}} \times }\underbrace{\sum_{j=1}^{k} n_{i,j}(n_{i,j}-1)}_{\textnormal{abs. pairwise agr. for item }i} \end{equation*}}
\end{block}

\begin{block}<5->{Normalized observed agreement for all items}
\begin{equation*}
	P_o = \frac{1}{N} \sum_{i=1}^N \hat P_i
\end{equation*}

(this is just the arithmetic mean, a.k.a. average)
\end{block}

\begin{tikzpicture}[remember picture,overlay]
\node[at=(current page.north east),font=\tiny,anchor=north east,align=center] {
\begin{tabular}{lcccc}
\toprule
Word & A & B & C & D \\
\midrule
1 & 2 & 1 &   &   \\
2 &   &   & 1 & 2 \\
3 & 1 & 1 &   & 1 \\
4 & 3 &   &   &   \\
\bottomrule
\end{tabular}\\
$N = 4, k = 4, n = 3$
};
\end{tikzpicture}

\end{frame}

\begin{frame}{Inter-Annotator Agreement}{Expected Agreement}
\begin{outline}
\1 Probability that category $j$ gets selected (by one annotator)
\begin{equation}
	\uncover<3->{p_j =} \only<2->{\underbrace{\frac{1}{nN}}_{\textnormal{Possible events (all annotations)}} \times}\underbrace{\sum_{i=1}^N n_{i,j}}_{\textnormal{positive events (=annotations with } j \textnormal{)}}\nonumber
\end{equation}
\1<4-> Probability that two annotators select category $j$
\begin{equation*}
	p_j \times p_j = p_j^2
\end{equation*}
\1<5-> Probability that two annotators are in agreement (over all categories):
\begin{equation}
P_e = \sum_{j=1}^k p_j^2\nonumber
\end{equation}
\end{outline}
\begin{tikzpicture}[remember picture,overlay]
\node[at=(current page.north east),font=\tiny,anchor=north east,align=center] {
\begin{tabular}{lcccc}
\toprule
Word & A & B & C & D \\
\midrule
1 & 2 & 1 &   &   \\
2 &   &   & 1 & 2 \\
3 & 1 & 1 &   & 1 \\
4 & 3 &   &   &   \\
\bottomrule
\end{tabular}\\
$N = 4, k = 4, n = 3$
};
\end{tikzpicture}
\end{frame}

\begin{frame}{Fleiss' Kappa \parencite{Fleiss:1971aa}}

\begin{eqnarray*}
p_j & & \textnormal{Probability for } j \\
\begin{ensuremath}
\hat P_i
\end{ensuremath} & & \textnormal{Observed agreement for } i\\
\only<2->{P_o & = & \frac{1}{N} \sum_{i=1}^N \hat P_i \\}
\only<3->{P_e & = & \sum_{j=1}^k p_j^2 \\}
\only<4->{\kappa & = & \frac{P_o-P_e}{1-P_e}}
\end{eqnarray*}

\begin{outline}
\only<5->{
\1 $P_o - P_e$: Tatsächlich erreichtes, nicht-zufälliges Agreement
\1 $1 - P_e$: Maximal erreichbares, nicht-zufälliges Agreement
}
\only<6->{
\1 $-\infty < \kappa < 1$: Je höher desto besser
\2 Extremfälle?
}
\end{outline}
\end{frame}

\appendix

\section{Fleiss' Kappa}

\frame{\sectionpage}

\begin{frame}{Inter-Annotator Agreement}
\begin{outline}
\1 Warum quantitativ messen?
\2 Weil wir vergleichen wollen, über verschiedene Konfigurationen hinweg
\3 (5 oder 6 Kategorien, 2 oder 3 AnnotatorInnen, 10 oder 20 Instanzen)
\pause
\1 Wie messen?
\2 Problem: Wir wissen nicht was richtig ist
\3 IAA ist ausschließlich eine Aussage über die Übereinstimmung, nicht über die Korrektheit
\pause
\2 Verfahren das für beliebige Zahlen von Kategorien, AnnotatorInnen oder Instanzen funktioniert
\2 Wie oft haben zwei AnnotatorInnen gleich gewählt? (paarweise Übereinstimmung)
\end{outline}
\end{frame}

\begin{frame}[fragile]{Inter-Annotator Agreement}{Datenstrukturen}

\centering

\begin{tabular}{lccc}
\toprule
Word & A1 & A2 & A3 \\
\midrule
Zwei & ART & ART & CARD \\
Hunde & NN & NNS & NNS \\
bellen & VVFIN & VVINF & VVFINX \\
. & \$. & \$. & \$. \\
\bottomrule
\end{tabular}

\pause
\vspace{1em}
$\Downarrow$ Konversion $\Downarrow$
\vspace{1em}

\begin{tabular}{lcccccccccccc}
\toprule
Word & ART & CARD & NN & NNS & VVFIN & VVINF & VVFINX & \$. \\
\midrule
Zwei & 2 & 1 \\
Hunde & & & 1 & 2 \\
bellen & & & & & 1 & 1 & 1 \\
. & & & & & & & & 3 \\
\bottomrule
\end{tabular}

\end{frame}

\begin{frame}[fragile]{Inter-Annotator Agreement}{Paarungen}

\begin{tabular}{lcccccccccccc}
\toprule
Word & ART & CARD & NN & NNS & VVFIN & VVINF & VVFINX & \$. \\
\midrule
Zwei & 2 & 1 \\
Hunde & & & 1 & 2 \\
bellen & & & & & 1 & 1 & 1 \\
. & & & & & & & & \alert<3>{3} \\
\bottomrule
\end{tabular}

\begin{outline}
\1 Wieviele paarweise Übereinstimmungen gibt es (absolut gemessen)?
\pause
\2 $1 + 1 + 0 + 3 = 5$
\pause
\1 Warum 3? Binomialkoeffizient!
\end{outline}

\end{frame}

\begin{frame}{Binomialkoeffizient / `n choose k'}{Anzahl möglicher $k$-elementiger Teilmengen aus $n$ Dingen}
\begin{equation*}
{n\choose k} = \frac{n!}{k!(n-k)!}
\end{equation*}
\pause
Fakultät:
\begin{equation*}
n! = n(n-1)(n-2)(n-3)\cdots 1
\end{equation*}
\pause
Für $k=2$:
\begin{eqnarray*}
{n\choose 2} & = & \frac{n!}{(n-2)!2!} \pause = \frac{1}{2}\frac{n!}{(n-2)!} \\
\pause
& = & \frac{1}{2}\frac{n(n-1)(n-2)(n-3)\cdots 1}{(n-2)(n-3)\cdots 1} \\
\pause
& = & \frac{1}{2}n(n-1)	\\
\end{eqnarray*}
\end{frame}

\begin{frame}{Binomialkoeffizient / `n choose k'}{Anzahl möglicher $k$-elementiger Teilmengen aus $n$ Dingen}
Reihenfolge: Sind $(a,b)$ und $(b,a)$ das gleiche Paar?
\begin{outline}
\1 Ja: $\frac{1}{2}n(n-1)$
\1 Nein: $n(n-1)$
\end{outline}
\end{frame}

\begin{frame}{Inter-Annotator Agreement}{Paarungen}
\begin{table}
\begin{tabular}{rr}
\toprule
$n$ & ${n\choose 2}$ \\
\midrule
1 & 0 \\
2 & 1 \\
3 & 3 \\
4 & 6 \\
5 & 10 \\
10 & 45 \\
100 & 4.950 \\
\bottomrule
\end{tabular}
\caption{$n\choose 2$ Werte für steigende $n$}
\end{table}
\end{frame}

\begin{frame}{Inter-Annotator Agreement}{Problems}
\begin{columns}[T]
\begin{column}{.45\paperwidth}
Situation 1:

\begin{tabular}{lcccc}
\toprule
Word & A & B & C & D \\
\midrule
1 & 2 & 1 &   &   \\
2 &   &   & 1 & 2 \\
3 & 1 & 1 &   & 1 \\
4 & 3 &   &   &   \\
\bottomrule
\end{tabular}
\begin{outline}
\1 3 annotators (=n)
\1 5 pairwise agreements
\end{outline}
\end{column}

\pause

\begin{column}{.45\paperwidth}
Situation 2:

\begin{tabular}{lcccc}
\toprule
Word & A & B & C & D \\
\midrule
1 & 2 & 1 & 1 &   \\
2 &   & 1 & 1 & 2 \\
3 & 1 & 1 & 1 & 1 \\
4 & 3 & 1 &   &   \\
\bottomrule
\end{tabular}
\begin{outline}
\1 4 annotators (=n)
\1 5 pairwise agreements
\end{outline}
\end{column}
\end{columns}
\pause
\vspace{2em}
How much worse is Situation 2 compared to 1?\\
$\rightarrow$ Scaling!
\end{frame}

\begin{frame}{Inter-Annotator Agreement}{Scaling}
\begin{outline}
\1 Sometimes, values have different scales
\2 i.e., different min and max values
\1 Scaling: Apply a function to values so that they are comparable
\2 Simplest way: Divide by the (theoretical) maximum
\pause
\1 What's the theoretical maximum here?
\2 If all annotators agree: ${n\choose 2} = \frac{1}{2}n(n-1)$
\pause
\1 Scaling $\simeq$ Normalization
\end{outline}
\end{frame}

\begin{frame}
\begin{columns}[T]
\begin{column}{.45\paperwidth}
Situation 1:

\begin{tabular}{lcccc}
\toprule
Word & A & B & C & D \\
\midrule
1 & 2 & 1 &   &   \\
2 &   &   & 1 & 2 \\
3 & 1 & 1 &   & 1 \\
4 & 3 &   &   &   \\
\bottomrule
\end{tabular}
\begin{outline}
\1 3 annotators (=n)
\1 5 pairwise agreements
\1 Scaled: $\frac{5}{4{3\choose 2}} = \frac{5}{4\times 3} = \frac{5}{12} = 0.416$
\end{outline}
\end{column}

\begin{column}{.45\paperwidth}
Situation 2:

\begin{tabular}{lcccc}
\toprule
Word & A & B & C & D \\
\midrule
1 & 2 & 1 & 1 &   \\
2 &   & 1 & 1 & 2 \\
3 & 1 & 1 & 1 & 1 \\
4 & 3 & 1 &   &   \\
\bottomrule
\end{tabular}
\begin{outline}
\1 4 annotators (=n)
\1 5 pairwise agreements
\1 Scaled: $\frac{5}{4{4\choose 2}} = \frac{5}{4 \times 6} = \frac{5}{24} = 0.208$
\end{outline}
\end{column}
\end{columns}
\end{frame}

\begin{frame}{Inter-Annotator Agreement}{Observed Agreement}

\begin{block}{Normalized observed agreement for item $i$}
Problem: $k$ categories, $n$ annotators, $N$ items 
\uncover<2->{\begin{equation*}\only<4->{\hat P_i = }\uncover<3->{\alert<3>{\underbrace{\frac{1}{n(n-1)}}_{\textnormal{Scaling for annotators}}} \times }\underbrace{\sum_{j=1}^{k} n_{i,j}(n_{i,j}-1)}_{\textnormal{abs. pairwise agr. for item }i} \end{equation*}}
\end{block}

\begin{block}<5->{Normalized observed agreement for all items}
\begin{equation*}
	P_o = \frac{1}{N} \sum_{i=1}^N \hat P_i
\end{equation*}

(this is just the arithmetic mean, a.k.a. average)
\end{block}

\begin{tikzpicture}[remember picture,overlay]
\node[at=(current page.north east),font=\tiny,anchor=north east,align=center] {
\begin{tabular}{lcccc}
\toprule
Word & A & B & C & D \\
\midrule
1 & 2 & 1 &   &   \\
2 &   &   & 1 & 2 \\
3 & 1 & 1 &   & 1 \\
4 & 3 &   &   &   \\
\bottomrule
\end{tabular}\\
$N = 4, k = 4, n = 3$
};
\end{tikzpicture}

\end{frame}

\begin{frame}{Large Operators}
\begin{equation*}
\sum_{i=0}^n f(i) = f(0) + f(1) + f(2) + \cdots f(n)
\end{equation*}
\pause
\begin{example}
\begin{eqnarray*}
\sum_{i=0}^4 i^5 & = & 0^5 + 1^5 + 2^5 + 3^5 + 4^5 = 1300\\
\pause
\prod_{i=1}^3 (5i+1) & = & (5(1)+1) \times (5(2)+1) \times (5(3)+1) \\
& = & 6+11+16 = 33\\
\end{eqnarray*}	
\end{example}
\end{frame}

% \begin{frame}{Inter-Annotator Agreement}{Observed Agreement}
%
% \begin{outline}
% \1 Input table \\\begin{tabular}{rcc}
% 		\toprule
% 		Categories $j$ $\rightarrow$ & A & B \\
% 		\midrule
% 		\multirow{3}{*}{\begin{turn}{90}Items $i$\end{turn}}
% 		& 3 & 0 \\
% 		& 2 & 1 \\
% 		& 0 & 3 \\
% 		\bottomrule
% 	\end{tabular}
% \1 Normalized observed agreement for item $i$
% \2 $\hat P_i = \frac{1}{n(n-1)} \times \sum_{j=1}^{k} n_{i,j}(n_{i,j}-1)$
% \1 Normalized observed agreement for all items
% \2 $P_o = \frac{1}{N}\times\sum_{i=1}^N \hat P_i$
% \end{outline}
%
% \end{frame}

\begin{frame}{Inter-Annotator Agreement}{Expected Agreement}
\begin{columns}[T]
\begin{column}{.45\paperwidth}
Situation 1:

\begin{tabular}{lcccc}
\toprule
Word & A & B & C & D \\
\midrule
1 & 2 & 1 &   &   \\
2 &   &   & 1 & 2 \\
3 & 1 & 1 &   & 1 \\
4 & 3 &   &   &   \\
\bottomrule
\end{tabular}
\begin{outline}
\1 $n=3$ annotators
\1 $k=4$ categories
\1 5 pairwise agreements
\end{outline}
\end{column}

\begin{column}{.45\paperwidth}
Situation 2:

\begin{tabular}{lccc}
\toprule
Word & A & B & C \\
\midrule
1 & 2 & 1 &   \\
2 & 2 &   & 1 \\
3 & 1 & 1 & 1 \\
4 & 3 &   &   \\
\bottomrule
\end{tabular}
\begin{outline}
\1 $n=3$ annotators
\1 $k=3$ categories
\1 5 pairwise agreements
\end{outline}
\end{column}
\end{columns}
\vspace{2em}
What situation had the better agreement?
How much better?
\end{frame}

\begin{frame}{Inter-Annotator Agreement}{Expected Agreement}
\begin{outline}
\1 How likely is it (in general) that category $A$ is selected?
\end{outline}

\begin{block}<2->{Excurs: Relative Frequencies}
\begin{columns}[onlytextwidth,T]
\begin{column}{\dimexpr\linewidth-30mm-5mm}
\begin{outline}
\1<2-> How often do you win a game?
\1<3-> $\frac{\textnormal{Number of positive events (= wins)}}{\textnormal{Number of total events(= games played)}}$
\1<4-> $p(\textnormal{"selecting category A"}) =\ ?$
\end{outline}
\end{column}
\end{columns}
\begin{outline}
\1<5-> Relative frequency is an \emph{estimate} of the probability
\2 Probability: Theoretical concept
\2 Relative frequency: Result of actual experiments
\2 Assumption: The more experiments I do, the more similar is the relative frequency to the probability	
\end{outline}
\end{block}
\end{frame}

\begin{frame}{Inter-Annotator Agreement}{Expected Agreement}
	
\begin{columns}[T]
\begin{column}{.45\paperwidth}
Situation 1:

\begin{tabular}{lcccc}
\toprule
Word & A & B & C & D \\
\midrule
1 & 2 & 1 &   &   \\
2 &   &   & 1 & 2 \\
3 & 1 & 1 &   & 1 \\
4 & 3 &   &   &   \\
\bottomrule
\end{tabular}
\end{column}

\begin{column}{.45\paperwidth}
Situation 2:

\begin{tabular}{lccc}
\toprule
Word & A & B & C \\
\midrule
1 & 2 & 1 &   \\
2 & 2 &   & 1 \\
3 & 1 & 1 & 1 \\
4 & 3 &   &   \\
\bottomrule
\end{tabular}
\end{column}
\end{columns}

\begin{outline}
\1 $p(\textnormal{"selecting category A"}) = \frac{\textnormal{Number of positive events}}{\textnormal{Number of possible events}}$
\2 Positive events: How often category A has been selected
\2 Possible events: How many decisions have been made
\1 $p(\textnormal{"selecting category A"}) = $\\
\2 Situation 1: $\frac{6}{12}$
\2 Situation 2: $\frac{8}{12}$
\end{outline}
\end{frame}

\begin{frame}{Inter-Annotator Agreement}{Expected Agreement}
\begin{outline}
\1 Probability that category $j$ gets selected (by one annotator)
\begin{equation}
	\uncover<3->{p_j =} \only<2->{\underbrace{\frac{1}{nN}}_{\textnormal{Possible events (all annotations)}} \times}\underbrace{\sum_{i=1}^N n_{i,j}}_{\textnormal{positive events (=annotations with } j \textnormal{)}}\nonumber
\end{equation}
\1<4-> Probability that two annotators select category $j$
\begin{equation*}
	p_j \times p_j = p_j^2
\end{equation*}
\1<5-> Probability that two annotators are in agreement (over all categories):
\begin{equation}
P_e = \sum_{j=1}^k p_j^2\nonumber
\end{equation}
\end{outline}
\end{frame}



\begin{frame}{Fleiss' Kappa \parencite{Fleiss:1971aa}}

\begin{eqnarray*}
p_j & & \textnormal{Probability for } j \\
\begin{ensuremath}
\hat P_i
\end{ensuremath} & & \textnormal{Observed agreement for } i\\
\only<2->{P_o & = & \frac{1}{N} \sum_{i=1}^N \hat P_i \\}
\only<3->{P_e & = & \sum_{j=1}^k p_j^2 \\}
\only<4->{\kappa & = & \frac{P_o-P_e}{1-P_e}}
\end{eqnarray*}

\begin{outline}
\only<5->{
\1 $P_o - P_e$: Tatsächlich erreichtes, nicht-zufälliges Agreement
\1 $1 - P_e$: Maximal erreichbares, nicht-zufälliges Agreement
}
\only<6->{
\1 $-\infty < \kappa < 1$: Je höher desto besser
\2 Extremfälle?
}
\end{outline}
\end{frame}

\end{document}